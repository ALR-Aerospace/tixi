
\chapter{Using TIXI}\label{usingTixi}

\section{Usage Example}\label{Usage Example}

The following example assume the XML example from section XPath Examples to be stored in a file named \textit{ju.xml}.

\subsection{Retrieve the position of the first wing}

\begin{verbatim}
 #include "tixi.h"

 char* xmlFilename = "ju.xml";
 TixiDocumentHandle handle = -1;
 char * elementPath = "/plane/wings/wing[1]";
 char* attributeName = "position";

 char* attributeValue;

 tixiOpenDocument( xmlFilename, &handle );
 tixiGetTextAttribute( handle, elementPath, attributeName, &attributeValue );
 tixiCloseDocument( handle );
\end{verbatim}




\subsection{Retrieve x value of the coordinate origin}

\begin{verbatim}
 #include "tixi.h"

 char* xmlFilename = "ju.xml";
 TixiDocumentHandle handle = -1;
 char * elementPath = "/plane/coordinateOrigin/x";
 double x = 0.;

 tixiOpenDocument( xmlFilename, &handle );
 tixiGetDoubleElement( handle, elementPath, &x );
 tixiCloseDocument( handle );
\end{verbatim}



\section{Notes for Fortran programming}

The Fortran interface is implemented by calls to subroutines.
It assumes the follwing mapping for the basic types:
real is \textsl{real*8} corresponds double
integer is \textsl{integer*4} corresponds to int
character corresponds char
Character strings are to be passed as variables of type \textsl{character*N}. If a string is returned by a subroutine call the variable holding the result must be large enough to hold the result. Otherwise the result is truncated and the return code is set to \textsl{STRING\_TRUNCATED}.

\textbf{NOTE:} In view of these restrictions an implementation using character arrays should be considered.
The return codes returned in the last argument corresponds to their position in ReturnCode starting with 0. A routine will be supplied to directly access the meaning of a return code as a string.
When the C interface requires to pass a NULL-pointer, e.g. to choose the default format \textsl{"\%g"} when writing floating point elements, the respective argument in the Fortran interfaces is the empty string constant "". This is the only way to represent a string of length zero. Passing a variable with all characters set to "" will, via the interface transformed into an emtpy C-string "\\0", which is of length 1, and not to a NULL-pointer. 
